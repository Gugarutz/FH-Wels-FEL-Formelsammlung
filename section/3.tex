
\section{Biegungen, Verzerrungen und Energien (?)}

\paragraph{ Verzerrung eines Balkens}
	$ U = \ds
	      \int_{x_1}^{x_2} \dfrac{N^2}  {2\ E A}   \dd x
	    + \int_{x_1}^{x_2} \dfrac{M_b^2}{2\ E J_b} \dd x 
	    + \int_{x_1}^{x_2} \dfrac{M_T^2}{2\ G J_T} \dd x 
	    + \int_{x_1}^{x_2} \dfrac{Q^2}  {2\ G A_s} \dd x $

\paragraph{ Arbeitssatz}
	$ W = U \qquad 
		W = \dfrac{1}{2}\ F\ w_F$

\paragraph{ Castigliano:}
	$ w_f  = \dfrac{\partial U}{\partial F_i} \qquad w_H = \dfrac{\partial U}{\partial H} $
	
\paragraph{ Menabrea:}
	$ \dfrac{\partial U}{\partial X_i} = 0 $
		
		\vskip 3pt
	$ X_i $\dots Auflagekraft; statisch unbestimmte, innere Kraft
	
	\textbf{Äußere/Innerlich statische Bestimmtheit:}
	
	2-dimensional:\quad $ N = Z + R - 3\ K $
		\hfil 3-dimensional:\quad $ N = Z + R - 6\ K $ \hfil
	
	$ Z $\dots Zwangskräfte
		\qquad $ R $\dots Reaktionskräfte (Lagerkräfte)
		\qquad $ K $\dots Körper
	
\paragraph{ Ritz:} \quad siehe S.176 im Skript
	
	$ \vec{u} \approx \tilde{u}_{(x,y,z)} = \ds\sum_k a_k\ \vec{\varphi}_{k(x,y,z)} $
		\qquad $ \vec{\varphi}_{k(x,y,z)} $\dots Ritz'sche Ansatzfunktion
		\qquad $ a_k $\dots Koeffizienten
	
	$ V \approx \tilde{V}_{(\tilde{u})} = \tilde{V}_{(a_k)} $
		\qquad $ \dfrac{\partial \tilde{V}_{(a_k)}}{\partial a_k} = 0 \rightarrow a_k $
